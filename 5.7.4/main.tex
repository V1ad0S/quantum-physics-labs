\documentclass[reprint, nofootinbib, 10pt]{revtex4-2}

\usepackage[T2A]{fontenc}			% кодировка
\usepackage[utf8]{inputenc}			% кодировка исходного текста
\usepackage[english,russian]{babel}	% локализация и переносы

\usepackage{amsmath,amsfonts,amssymb,amsthm,mathtools}

\usepackage[usenames]{color}
\usepackage{colortbl}
\usepackage{indentfirst} %Красная строка
\usepackage{hyperref}

\usepackage{booktabs}
\usepackage{graphicx}  % Для вставки рисунков
\graphicspath{{images/}{graphs/}}  % папки с картинками

\renewcommand*{\thefootnote}{\alph{footnote}}

\begin{document}

\title{Исследование поглощения вторичного космического излучения в веществе}
\author{Илларионов Владислав}
\affiliation{группа Б04-855}

\maketitle


\section*{Введение}

В результате взаимодействия протонов первичного излучения с ядрами атомов воздуха в
атмосфере Земли происходит расщепление ядер и рождение $\pi^{\pm}$-мезонов и
$\pi^0$-мезонов. Распад заряженных $\pi^{\pm}$-мезонов приводит к образованию жесткой
компоненты вторичного космического излучения, а распад нейтральных пионов $\pi^0$~--~к
образованию мягкой (электронно-фотонной) компоненты.

В данной работе по измерениям зависимости интенсивности космического излучения в
лаборатории от толщины поглотителя (пластины свинца) определяются эффективные длины
поглощения мягкой и жесткой компонент космики, а также абсолютные значения их вертикальных
интенсивностей.


\section*{Теоретическая часть}

Космические лучи~--~поток частиц высокой энергии, преимущественно протонов, приходящих
на Землю из мирового пространства (первичное излучение), а также рожденное ими в атмосфере
Земли из-за взаимодействия с атомными ядрами вторичное излучение. Результатом этого
взаимодействия преимущественно происходит рождение пионов, распад которых приводит к
образованию мюонов и гамма-квантов:

\begin{eqnarray*}
    \pi^+ &&\rightarrow \mu^+ + \nu_{\mu}, \\
    \pi^- &&\rightarrow \mu^- + \nu_{\mu}, \\
    \pi^0 &&\rightarrow \gamma + \gamma
\end{eqnarray*}

Жесткая компонента в основном состоит из мюонов, которые не участвуют в ядерных (сильных)
взаимодействиях и практически не теряют своей энергии за счет тормозного излучения. Их
энергия тратится только на ионизацию вещества. Ионизационные потери релятивистских мюонов
слабо зависят от состава вещества и фактически определяются лишь поверхностной плотностью
поглотителя.

В отличие от мюонов, потеря энергии высокоэнергичными фотонами обусловлена процессом
рождения пар в веществе, а электроны теряют свою энергию за счет тормозного излучения.
Образовавшиеся фотоны с большой вероятностью снова рождают электрон-позитронные пары.
Так, быстро образуется лавина. С увеличением толщины поглотителя, все больше выбывает
электронов и позитронов, в следствие чего уменьшается количество регистрируемых частиц. 


\section*{Экспериментальная установка}

Основой измерительной установки является телескоп, ориентированный вертикально и
отбирающий для регистрации лишь те частицы, которые приходят в определенном направлении
внутри телесного угла, определяемого геометрией детекторов.

\textcolor{red}{TODO: разобраться, описание какой установки использовать}


\section*{Методика измерения}

Количество частиц $N$, регистрируемых счетчиком, как дискретная случайная величина, может
быть описана распределением Пуассона ($\lambda = N$). Тогда относительная погрешность
определения $N$ будет равна:

\[ \varepsilon = \dfrac{\sqrt{N}}{N} = \dfrac{1}{\sqrt{N}} \]

Производится серия измерений зависимости количества регистрируемых импульсов от толщины
поглотителя за определенное время. Замеры ведутся одновременно на двух установках, после
чего производится анализ полученных данных. Для каждого значения толщины поглотителя
берется среднее значение количества импульсов в секунду.


\end{document}